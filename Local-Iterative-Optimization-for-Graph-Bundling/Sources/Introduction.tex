%!TEX root = ../Thesis.tex

%%%%%%%%%%%%%%%%%%%%%%%%%%%%%%%%%%%%%%%%%%%%%%%%%%%%%%%%%%%%%%%%%%%%%%%%
\chapter{Introduction}
\label{sec:introduction}
%%%%%%%%%%%%%%%%%%%%%%%%%%%%%%%%%%%%%%%%%%%%%%%%%%%%%%%%%%%%%%%%%%%%%%%%

Today complex relations are usually visualized as graphs to make them readable and straightforward to understand, as in the case of tube maps, graphs of migration and flight patterns, network connections, chip design, and many more. 
The difficulty with that practice is that current datasets are massive, and the more data is visualized, the more challenging it gets to comprehend the resulting graph. This predicament led to the development of edge bundling techniques. These enable graphs to become better readable and uncover previously concealed patterns. In this thesis, we want to address two problems with the existing edge bundling approaches and present our novel idea to mitigate those.

The first problem inherent in many current bundling techniques is the absence of a mathematical structure. Most methods begin bundling without indicating where the bundling should go. For instance, the Edge-Path bundling approach by Wallinger et al. \cite{wallinger_edge-path_2022} bundles long edges along short ones. This method delivers a visually pleasing result, though it is impossible to anticipate the outcome. This thesis addresses this issue by utilizing the Steiner tree as a mathematical structure for our bundling algorithm. A Steiner tree links nodes of graphs with the shortest Euclidean distance and is thus even shorter than a minimal spanning tree. This result is achievable by adding so-called Steiner points to the graph \cite{brazil_optimal_2015} — more on Steiner trees in \autoref{sec:steinertrees}.

The second problem is ambiguities like node-edge overlaps or edge crossings - more on them in \autoref{sec:bundling_techniques}. These ambiguities are inherent in all graphs, but edge-bundled ones are more prone to them. The Steiner tree has some advantages regarding the ambiguity of the bundling, as the B\'{e}zier curve can use the nodes as control points. This visualization leads to more visible nodes, reducing the chances of node-edge overlaps. Additionally, the Steiner points can have a maximum of three edges, and the gradient between them cannot get smaller than 120° degrees \cite{brazil_optimal_2015}, which again aids in the readability of the bundling \cite{huang_effects_2008}.

The difficulty with Steiner trees is that they are NP-complete; thus, it is unattainable to compute a solution in polynomial time \cite{ljubic_solving_2021}. For this reason, we selected a Physarum Polycephalum approximation algorithm, first described by \cite{tero_physarum_2006}, as an altered membrane computing approach by \cite{liu_physarum_2015} to compute a Euclidean Steiner tree from our original graph data. This approximation is viable because the mass of the Physarum slime mold is proven to solve the Steiner tree problem provided with sufficient time \cite{zhang_improved_2014}. The decision to utilize a Physarum-inspired algorithm enables straightforward parallel processing, a dynamic way of computing Steiner trees, and a good groundwork for more advanced applications that we will discuss in \autoref{sec:conclusion}.

The following section reviews related works on edge bundling approaches, Physarum machines, and Steiner trees. 
In the third section, we lay the fundamentals and define our terminology for understanding graphs, edge bundling, Physarum polycephalum, and Steiner trees. A firm comprehension of these subjects is crucial for reading this thesis. The fourth paragraph elaborates on how we archived our results by listing the approaches and modifications we implemented. We discuss the result in the fifth section and compare them against two other edge bundling methods using different quality metrics. Lastly, we discuss the consequences and limitations of our approach and discuss future work.

The contribution of this work is a novel edge bundling approach that utilizes a Physarium approximation of Steiner trees as a routing structure for graph paths. The resulting drawings have a high degree of bundling and reduce the number of node-edge overlaps.