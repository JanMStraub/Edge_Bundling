%!TEX root = ../Thesis.tex

%%%%%%%%%%%%%%%%%%%%%%%%%%%%%%%%%%%%%%%%%%%%%%%%%%%%%%%%%%%%%%%%%%%%%%%%
\chapter{Conclusion}
\label{sec:conclusion}

This work introduced a novel edge bundling technique based on a Physarium approximation of a Steiner tree. This Steiner tree is then used as a routing structure for graph paths. We evaluated and compared our algorithm using the quality metrics: ink reduction, distortion, and ambiguity. These metrics are computed on graph drawings of synthetic data we generated for testing pursues. The resulting illustrations consist of dense graphs that reduce node-edge overlaps, make reading easier, and make it more straightforward to spot areas of interest. Our algorithm has a worst-case complexity of $O\left(T_0 \cdot \left(n \cdot e + T_i\left(n^3\right)\right)\right)$ which is enough for small-scale synthetic graphs, but not enough to be used with real-world data, mainly because graph bundling becomes necessary if the graph data it too large to comprehend. 

The metrics comparison showed us that our approach has a reasonable degree of bundling and outperforms the other two techniques. This, however, comes at the cost of distortion and ambiguity. Nevertheless, the excellent ink reduction value is a promising sign that with a better approximation algorithm, our approach could be used to declutter graphs and isolate regions of interest. 

For us, future work could lead in two directions. The first would be to improve performance by using a GPU-based implementation to solve the matrices or abandoning the Physarium approximation and instead employ GeoSteiner \cite{winter_euclidean_1997}, or even better SCIP-Jack \cite{RehfeldtKoch2023} for a fast Steiner tree approximation. The other direction would be to utilize the Physarium-inspired approach to a Euclidean Steiner tree by Hsu \cite{hsu_physarum-inspired_2022}. With this, it should be possible to bundle real-world graph data.

To conclude: Our approach suffers from significant drawbacks caused by the slow approximation calculation, which hindered us from testing real-world graph data and, in turn, evaluating the usefulness of our method. Unfortunately, we discovered too late that our optimizations could not improve the approximation enough to become functional. 
Nevertheless, we still think it is a promising approach that could become feasible by utilizing the above-mentioned future work. 
%%%%%%%%%%%%%%%%%%%%%%%%%%%%%%%%%%%%%%%%%%%%%%%%%%%%%%%%%%%%%%%%%%%%%%%%