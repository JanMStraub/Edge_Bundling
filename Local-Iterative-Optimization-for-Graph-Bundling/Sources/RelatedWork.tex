%!TEX root = ../Thesis.tex

%%%%%%%%%%%%%%%%%%%%%%%%%%%%%%%%%%%%%%%%%%%%%%%%%%%%%%%%%%%%%%%%%%%%%%%%
\chapter{Related Work}
\label{sec:relatedWork}
%%%%%%%%%%%%%%%%%%%%%%%%%%%%%%%%%%%%%%%%%%%%%%%%%%%%%%%%%%%%%%%%%%%%%%%%

As noted above, we separated this paragraph into the three areas linked to our topic: Edge bundling, Physarium-related work, and Steiner trees. We outline some well-known edge bundling techniques in the \autoref{sec:edgeBundling}. In \autoref{sec:physariumPolycephalum}, the focus is on research connected to Physarium Polycephalum simulations and Steiner trees. The section delivers a glance past computer science, and we present works that specifically utilize the slime mold to approximate Steiner trees. At last, we discuss articles that employ other strategies to compute Steiner trees.

%%%%%%%%%%%%%%%%%%%%%%%%%%%%%%%%%%%%%%%%%%%%%%%%%%%%%%%%%%%%%%%%%%%%%%%%
\section{Edge Bundling}
\label{sec:edgeBundling}

One of the earliest edge bundling papers is by Newbery et al. \cite{newbery_edge_1989}, where he reduced the number of edge crossings by introducing edge nodes in convoluted areas. In Confluent Drawings by Dickerson et al. \cite{dickerson_confluent_2002}, they tried the same by visualizing a non-planar graph in a planar way. With their approach, they eliminated edge crossings by only bundling edges if a sub-graph was present. This restriction, in turn, also means that it is impossible to bundle all graphs; therefore, the use cases for this algorithm are limited. Our approach suffers the identical drawback, as bundling is only possible for graphs of a specific size. Other notable approaches that build upon Dickerson et al. are by Holten et al. \cite{holten_hierarchical_2006}, and the subsequent work by Hui et al. \cite{hui_train_2007} and more recently by Bach et al. \cite{bach_towards_2017}.
The size constraints of graph bundling approaches led van-der-Zwan et al. to their real-time bundling method CUBu (CUDA-based Universal Bundling) \cite{van_der_zwan_cubu_2016}. With this GPU-based approach, it is possible to bundle up to one million edges in real-time. The same performance delivers the KDEEB (kernel density estimation edge-bundling) method by Hurter et al. \cite{hurter_graph_2012}, which uses image sharpening on the density map of the graph. FFTEB (Fast Fourier transform edge bundling) \cite{lhuillier_ffteb_2017} should also be mentioned here, as it delivers on par performance to the before mentioned methods.

In Force-Directed Edge Bundling \cite{holten_force-directed_2009}, Holten et al. divide each edge into segments, and each component attracts others like a spring or magnet. This method eliminates the need for a hierarchy contained in the graph that previous methods required. A technique similar to our first idea is Swarm-based edge bundling for graph visualization by Pennekamp et al. \cite{pennekamp_swarm-based_2019}, as they use an ant simulation to bundle edges, the same as we used Physarium agents. A recent paper by Wallinger et al. \cite{wallinger_edge-path_2022} is based on the confluent drawing approach by Dickerson et al. \cite{dickerson_confluent_2002} but relaxes the restrictive constraints to archive a better clutter reduction. They do not need any support structure; instead, they use the shortest graph edges and route all longer ones along the short ones. According to Wallinger et al., this lack of a support structure is a benefit. However, we beg to differ because bundling methods with a clear structure benefit from anticipating the algorithm's outcome. For this reason, we decided to use the Steiner tree in our technique.

Another interesting approach that uses a clear structure comes from Lambert et al. in Winding Roads \cite{lambert_winding_2010}. The idea is to build a grid-graph support structure on the graph using a combination of a quad-tree and Voronoi subdivision. This structure can then be used to route the paths of the graph on its edges. The LSQT (Low-Stretch Quasi-Trees) approach by Vanderberg et al. \cite{vandenberg_lsqt_2020} leverages properties of low-stretch trees that allow it to declutter graphs efficiently. 

Finally, the paper by Lhuillier et al. \cite{lhuillier_state_2017} overviews the current state-of-the-art edge and trail bundling techniques.

%%%%%%%%%%%%%%%%%%%%%%%%%%%%%%%%%%%%%%%%%%%%%%%%%%%%%%%%%%%%%%%%%%%%%%%%
\section{Physarium Machines and Steiner Trees}
\label{sec:physariumPolycephalum}

The scientific interest in the Polycephalum slime mold began after the article Maze-solving by an amoeboid organism by Nakagaki et al. \cite{nakagaki_maze-solving_2000, nakagaki_path_2001} in the year 2000. They found that the slime mold can find the shortest path between two points in a labyrinth if provided with food sources. Tero et al. \cite{tero_physarum_2006} later described the mathematical basis of this behavior and used it to find the shortest road-network path. As the performance of these early algorithms made them unpractical for significant graph problems, Zhang et al. \cite{zhang_improved_2014} improved it by adding a new variable that denotes the energy used by a tube and therefore reducing the number of iterations needed to converge. These improvements continued with a paper by Liu et al. \cite{liu_physarum_2015}, which lowers the complexity and enables parallel calculation. This concept was initially used as a basis for our algorithm. Still, as Liu et al. relied on sensors that are used to calculate the edge cost, we chose to use a similar paper instead: Fast Algorithms Inspired by Physarium Polycephalum for Node Weighted Steiner Tree Problem with Multiple Terminals by Sun et al. \cite{sun_fast_2016}. He takes the same concept as Liu but changes how the node and edge costs are considered. For our purposes, we modified his approach by reducing computation time and enabling parallel computing.  

The cellular computation of Polycephalum cannot only find the shortest path between food sources, but it can also construct Voronoi diagrams and perform Delaunay triangulation \cite{shirakawa_simultaneous_2009}. Another application for the slime mold is the approximation of the Traveling Salesman Problem. In their paper, Jones et al. \cite{jones_computation_2013} trace the outline of a shrunken blob to get the TSP tour. 
In Characteristics of Pattern Formation and Evolution in Approximations of Physarum Transport Networks \cite{jones_characteristics_2010} Jones took an agent-based approach to simulate Polycephalum. The agent-based approach allowed him not to be limited by any graph structure and only placed nodes at the appropriate positions so the agents could find their path between them. His method was used as the prototype for our algorithm. The absence of a clear graph structure in the result made calculating a resulting plot too complicated.  

Other approaches to calculating the Steiner Tree were made by Byrka et al. \cite{byrka_steiner_2013}, which used iterative randomized rounding to reduce the approximation ratio of a Steiner tree. Chen \cite{chen_efficient_2018} improves upon Byrka et al.'s method and minimizes the approximation ratio further. The currently most efficient and exact algorithm for computing minimum Steiner trees is GeoSteine by Winter et al. They use linear programming and code written in C to archive a Steiner tree computation of a 50-terminal graph in less than 8 minutes \cite{winter_euclidean_1997}. Over the years, they improved their algorithm and are now on version 5.2. SCIP-Jack is an even further improved version that is currently the fastest solver \cite{RehfeldtKoch2023}.

Ljubi\'{c} offers in Solving Steiner trees \cite{ljubic_solving_2021} an overview of the recent advances in this research area.

%%%%%%%%%%%%%%%%%%%%%%%%%%%%%%%%%%%%%%%%%%%%%%%%%%%%%%%%%%%%%%%%%%%%%%%%